\documentclass[12pt,russian]{beamer}
\usepackage{mypres}
\usepackage[utf8]{inputenc}
\usepackage[T2A]{fontenc}
\usepackage[russian]{babel}
\newcommand{\relint}{\mathbf{relint}}
\newcommand{\inter}{\mathbf{int}}

\expandafter\def\expandafter\insertshorttitle\expandafter{%
  \insertshorttitle\hfill%
  \insertframenumber\,/\,\inserttotalframenumber}
\title[Семинар 9]{Методы оптимизации. \\
 Семинар 9. Условия оптимальности.}
\author{Александр Катруца}
\institute{Московский физико-технический институт,\\
Факультет Управления и Прикладной Математики} 
\date{31 октября 2016 г.}

\begin{document}
\begin{frame}
\maketitle
\end{frame}

\begin{frame}{Напоминание}
\begin{itemize}
\item Конус возможных направлений
\item Касательный конус
\item Острый экстремум
\end{itemize}
\end{frame}

\begin{frame}{Что такое условия оптимальности?}

\end{frame}

\begin{frame}{Условия оптимальности}

\end{frame}

\begin{frame}{Примеры}

\end{frame}

\begin{frame}{Резюме}
\begin{itemize}
\item Теорема Вейерштрасса
\item Условия оптимальности
\end{itemize}
\end{frame}

\end{document}