\documentclass[12pt,russian]{beamer}
\usepackage{mypres}
\usepackage[utf8]{inputenc}
\usepackage[english,russian]{babel}
\usepackage[T2A]{fontenc}
\newcommand{\relint}{\mathbf{relint}}
\newcommand{\inter}{\mathbf{int}}

\expandafter\def\expandafter\insertshorttitle\expandafter{%
  \insertshorttitle\hfill%
  \insertframenumber\,/\,\inserttotalframenumber}
\title[Семинар 7]{Методы оптимизации. \\
 Семинар 7. Субдифференциал.}
\author{Александр Катруца}
\institute{Московский физико-технический институт,\\
Факультет Управления и Прикладной Математики} 
\date{17 октября 2016 г.}

\begin{document}
\begin{frame}
\maketitle
\end{frame}

\begin{frame}{Напоминание}
\begin{itemize}
\item Выпуклая функция
\item Надграфик и множество подуровня функции
\item Критерии выпуклости функции
\item Неравенство Йенсена
\end{itemize}
\end{frame}

\begin{frame}{Мотивация}
\begin{block}{Зачем?}
Для непрерывной выпуклой функции $f$ важно знать такой вектор $\ba$, что
\vspace{-3mm} 
\[
f(\bx) - f(\by) \geq \langle \ba, \bx - \by \rangle
\]
\end{block}

\begin{itemize}
\item Если $f$ дифференцируема, то $\ba = \nabla f(\by)$.
\item Что делать, если $f$ недифференцируема?
\end{itemize}

\end{frame}

\begin{frame}{Определение}

\end{frame}

\begin{frame}{Полезные факты}

\end{frame}

\begin{frame}{Примеры}

\end{frame}

\begin{frame}{Резюме}

\end{frame}

\end{document}