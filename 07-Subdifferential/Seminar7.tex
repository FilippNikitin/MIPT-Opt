\documentclass[12pt,russian]{beamer}
\usepackage{mypres}
\usepackage[utf8]{inputenc}
\usepackage[english,russian]{babel}
\usepackage[T2A]{fontenc}
\newcommand{\relint}{\mathbf{relint}}
\newcommand{\inter}{\mathbf{int}}

\expandafter\def\expandafter\insertshorttitle\expandafter{%
  \insertshorttitle\hfill%
  \insertframenumber\,/\,\inserttotalframenumber}
\title[Семинар 7]{Методы оптимизации. \\
 Семинар 7. Субдифференциал.}
\author{Александр Катруца}
\institute{Московский физико-технический институт,\\
Факультет Управления и Прикладной Математики} 
\date{17 октября 2016 г.}

\begin{document}
\begin{frame}
\maketitle
\end{frame}

\begin{frame}{Напоминание}
\begin{itemize}
\item Выпуклая функция
\item Надграфик и множество подуровня функции
\item Критерии выпуклости функции
\item Неравенство Йенсена
\end{itemize}
\end{frame}

\begin{frame}{Мотивация}
\begin{block}{Зачем?}
Для непрерывной выпуклой функции $f$ важно знать такой вектор $\ba$, что
\vspace{-3mm} 
\[
f(\bx) - f(\by) \geq \langle \ba, \bx - \by \rangle
\]
\end{block}

\begin{itemize}
\item Если $f$ дифференцируема, то $\ba = \nabla f(\by)$.
\item Что делать, если $f$ недифференцируема?
\end{itemize}

\end{frame}

\begin{frame}{Определение}
\begin{block}{Субградиент}
Вектор $\ba$ называется субградиентов функции $f: X \rightarrow \bbR^n$ в точке $\bx$, если 
\vspace{-3mm}
\[
f(\by) - f(\bx) \geq \langle \ba, \by - \bx \rangle
\]
для всех $\by \in X$.
\end{block}

\begin{block}{Субдифференциал}
Множество субградиентов функции $f$ в точке $\bx$ называется субдифференциалом $f$ в $\bx$ и обозначается $\partial f(\bx)$.
\end{block}
\end{frame}

\begin{frame}{Полезные факты}
\begin{block}{Теорема Моро-Рокафеллара}
Пусть $f_i(\bx)$~--- выпуклые функции на выпуклых множествах $G_i, \; i = 1,\ldots,n$. 
Тогда, если $\bigcap\limits_{i=1}^n \text{ri } G_i \neq \varnothing$ то функция $f(\bx) = \sum\limits_{i=1}^n a_i f_i(\bx), \; a_i > 0$ имеет субдифференциал $\partial_G f(\bx)$ на множестве $G = \bigcap\limits_{i=1}^n G_i$ и $\partial_G f(\bx) = \sum\limits_{i=1}^n a_i \partial_{G_i} f_i(\bx)$. 
\end{block}

\begin{block}{Если функция~--- максимум}
Если $f(\bx) = \max\limits_{i=1,\ldots,m}(f_i(\bx))$, тогда $\partial_G f(\bx) = \text{Conv} (\bigcup\limits_{i \in \calJ(\bx)} \partial_G f_i(\bx))$, где $\calJ(\bx) = \{ i = 1,\ldots, m | f_i(\bx) = f(\bx) \}$
\end{block}
\end{frame}

\begin{frame}{Примеры}
\begin{itemize}
\item Модуль: $f(x) = |x|$
\item Норма $\ell_2$: $f(\bx) = \| \bx \|_2$
\item Скалярный максимум: $f(x) = \max(e^x, 1 - x)$
\item Векторный максимум: $f(\bx) = |\bc^{\T}\bx|$
\item $f(\bx) = |\bc^{\T}_1\bx| + |\bc^{\T}_2\bx|$
\end{itemize}
\end{frame}

\begin{frame}{Условный субдифференциал}

\end{frame}

\begin{frame}{Резюме}
\begin{itemize}
\item Субградиент
\item Субдифференциал
\item Условный субдифференциал
\item Методы вычислений
\end{itemize}
\end{frame}

\end{document}