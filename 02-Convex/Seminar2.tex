\documentclass[12pt,russian]{beamer}
\usepackage{mypres}
\usepackage[utf8]{inputenc}
\usepackage[english,russian]{babel}
\usepackage[T2A]{fontenc}

\expandafter\def\expandafter\insertshorttitle\expandafter{%
  \insertshorttitle\hfill%
  \insertframenumber\,/\,\inserttotalframenumber}
\title[Семинар 1]{Методы оптимизации. \\
Семинар 2. Выпуклые множества.}
\author{Александр Катруца}
\institute{Московский физико-технический институт,\\
Факультет Управления и Прикладной Математики} 
\date{12 сентября 2016 г.}
\begin{document}
\begin{frame}
\maketitle
\end{frame}

\begin{frame}{Напоминание}

\begin{itemize}
\item Предмет и задача курса
\item Общая постановка задачи математического программирования
\item Примеры задач оптимизации: 
\begin{itemize}
\item линейное программирование
\item метод наименьших квадратов
\item выпуклая оптимизация
\end{itemize}
\item Чем хороши выпуклые задачи?
\end{itemize}
\end{frame}

\begin{frame}{Афинное множество}

\end{frame}

\begin{frame}{Выпуклое множество}
\begin{block}{Выпуклое множество}
Множество $C$ называется выпуклым, если 
\[
\forall x_1, \; x_2 \in C, \theta \in [0, 1] \rightarrow \theta x_1 + (1 - \theta)x_2 \in C.
\]
Пустое множество и множество из одной точки также считаются выпуклыми.
\end{block}
Очевидные примеры:
\begin{itemize}
\item всё пространство $\bbR^n$
\item афинное множество
\item луч
\item отрезок
\end{itemize}
\end{frame}

\begin{frame}{Операции, сохраняющие выпуклость}
%Элипсоид: $\mathcal{E}(x_c, \bP, r) = \{ x \; | \; (x - x_c)^{\T}\bP^{-1} (x - x_c) \leq r \}$
\end{frame}

\begin{frame}{Примеры}
\begin{itemize}
\item Шар по норме в $\bbR^n$: $B(r, x_c) = \{ x \; | \; \| x - x_c \| \leq r \}$
\item Mножество симметричных положительно-определённых матриц: $\bS^n_+ = \{ \bX \in \bbR^{n \times n} \; | \; \bX^{\T} = \bX, \; \bX \succeq 0 \}$
\item $\{ \bX \in \bbR^{n \times n} \; | \; \Tr(\bX) = 0 \}$
\end{itemize}
\end{frame}



\begin{frame}{Конус}

\end{frame}

\begin{frame}{Примеры}

\end{frame}

\begin{frame}{Оболочки множеств}

\end{frame}

\begin{frame}{Примеры}

\end{frame}

\end{document}
