\documentclass[12pt,russian]{beamer}
\usepackage{mypres}
\usepackage[utf8]{inputenc}
\usepackage[english,russian]{babel}
\usepackage[T2A]{fontenc}

\expandafter\def\expandafter\insertshorttitle\expandafter{%
  \insertshorttitle\hfill%
  \insertframenumber\,/\,\inserttotalframenumber}
\title[Семинар 2]{Методы оптимизации. \\
Семинар 2. Выпуклые множества.}
\author{Александр Катруца}
\institute{Московский физико-технический институт,\\
Факультет Управления и Прикладной Математики} 
\date{12 сентября 2016 г.}
\begin{document}
\begin{frame}
\maketitle
\end{frame}

\begin{frame}{Напоминание}

\begin{itemize}
\item Предмет и задача курса
\item Общая постановка задачи математического программирования
\item Примеры задач оптимизации: 
\begin{itemize}
\item линейное программирование
\item метод наименьших квадратов
\item выпуклая оптимизация
\end{itemize}
\item Чем хороши выпуклые задачи?
\end{itemize}
\end{frame}

\begin{frame}{Аффинное множество}
\small
\begin{block}{Аффинное множество}
Множество $A$ называется аффинным, если для любых $x_1$, $x_2 \in A$ и $\theta \in \bbR$ точка $\theta x_1 + (1 - \theta)x_2 \in A$.
\end{block}
Примеры: $\bbR^n$, гиперплоскость, точка.
\begin{block}{Аффинная комбинация точек}
Пусть $x_1, \ldots, x_k \in G$, тогда точка $\theta_1 x_1 + \ldots + \theta_k x_k$ при $\sum\limits_{i=1}^k \theta_i = 1$ называется аффинной комбинацией точек $x_1,\ldots,x_k$.
\end{block}

\begin{block}{Аффинная оболочка точек}
Множество $\left\{ \sum\limits_{i=1}^k \theta_i x_i \; | \; x_i \in G, \sum\limits_{i=1}^k \theta_i = 1 \right\}$ называется аффинной оболочкой множества $G$ и обозначается \textbf{aff}(G).
\end{block}
\end{frame}

\begin{frame}{Утверждения}

\begin{block}{Утверждение 1}
Множество является аффинным тогда и только тогда, когда в него входят все аффинные комбинации его точек.
\end{block}

\begin{block}{Утверждение 2}
Множество является аффинным тогда и только тогда, когда его можно представить в виде $\{\bx | \bA\bx = \mathbf{b} \}$.
\end{block}
\end{frame}

\begin{frame}{Выпуклое множество}
\small
\begin{block}{Выпуклое множество}
Множество $C$ называется выпуклым, если 
\vspace{-4mm}
\[
\forall x_1, \; x_2 \in C, \theta \in [0, 1] \rightarrow \theta x_1 + (1 - \theta)x_2 \in C.
\vspace{-4mm}
\]
$\emptyset$ и $\{ x_0 \}$ также считаются выпуклыми.
\end{block}
Примеры: $\bbR^n$, афинное множество, луч, отрезок.
\begin{block}{Выпуклая комбинация точек}
Пусть $x_1, \ldots, x_k \in G$, тогда точка $\theta_1 x_1 + \ldots + \theta_k x_k$ при $\sum\limits_{i=1}^k \theta_i = 1, \;\theta_i \geq 0$ называется выпуклой комбинацией точек $x_1,\ldots,x_k$.
\end{block}

\begin{block}{Выпуклая оболочка точек}
Множество $\left\{ \sum\limits_{i=1}^k \theta_i x_i \; | \; x_i \in G, \sum\limits_{i=1}^k \theta_i = 1, \theta_i \geq 0 \right\}$ называется выпуклой оболочкой множества $G$ и обозначается \textbf{conv}(G).
\end{block}

\end{frame}

\begin{frame}{Операции, сохраняющие выпуклость}
\begin{itemize}
\item Пересечение любого (конечного или бесконечного) числа выпуклых множеств~--- выпуклое множество
\item Образ аффинного отображения выпуклого множества~--- выпуклое множество
\item Линейная комбинация выпуклых множеств~--- выпуклое множество
\item Декартово произведение выпуклых множеств~--- выпуклое множество
\end{itemize}
\end{frame}

\begin{frame}{Примеры}
\begin{itemize}
\item Полупространство: $\{ \bx | \ba^{\T} \bx \leq c \}$
\item Многоугольник: $\{ \bx | \bA\bx \preceq \mathbf{b}, \; \bC\bx = 0 \}$
\item Шар по норме в $\bbR^n$: $B(r, x_c) = \{ x \; | \; \| x - x_c \| \leq r \}$
\item Элипсоид: $\mathcal{E}(x_c, \bP, r) = \{ x \; | \; (x - x_c)^{\T}\bP^{-1} (x - x_c) \leq r \}$
\item Mножество симметричных положительно-определённых матриц: $\bS^n_+ = \{ \bX \in \bbR^{n \times n} \; | \; \bX^{\T} = \bX, \; \bX \succeq 0 \}$
\item $\{ \bX \in \bbR^{n \times n} \; | \; \Tr(\bX) = const \}$
\item Гиперболическое множество: $\{ \bx \in \bbR^n_+ \; | \; \prod\limits_{i=1}^n x_i \geq 1 \}$
\end{itemize}
\end{frame}

\begin{frame}{Конус}
\small
\begin{block}{Конус (выпуклый)}
Множество $C$ называется конусом (выпуклым конусом), если 
\vspace{-4mm}
\begin{equation*}
\begin{split}
& \forall x \in C, \theta \geq 0 \rightarrow \theta x \in C \\
& (\forall x_1, x_2 \in C, \theta_1, \theta_2 \geq 0 \rightarrow \theta_1 x_1 + \theta_2 x_2 \in C)
\end{split}
\end{equation*}
\vspace{-4mm}
\end{block}
Примеры: $\bbR^n$, афинное множество, проходящее через 0, луч.
\begin{block}{Коническая (неотрицательная) комбинация точек}
Пусть $x_1, \ldots, x_k \in G$, тогда точка $\theta_1 x_1 + \ldots + \theta_k x_k$ при $\theta_i \geq 0$ называется конической (неотрицательной) комбинацией точек $x_1,\ldots,x_k$.
\end{block}

\begin{block}{Коническая оболочка точек}
Множество $\left\{ \sum\limits_{i=1}^k \theta_i x_i \; | \; x_i \in G, \theta_i \geq 0 \right\}$ называется конической оболочкой множества $G$ и обозначается \textbf{cone}(G).
\end{block}
\end{frame}

\begin{frame}{Примеры}
\begin{itemize}
\item $\bS^n_+$
\item Нормальный конус: $\{ (\bx, t) \in \bbR^{n+1} \; | \; \| \bx \| \leq t \}$ 

Для $\ell_2$-нормы называется конусом второго порядка или Лоренцевым конусом
\item Конкретные примеры с числами
\end{itemize}
\end{frame}

\begin{frame}{Резюме}
\begin{itemize}
\item Аффинное множество
\item Выпуклое множество
\item Конус
\item Методы проверки свойств конкретных множеств
\end{itemize}
\end{frame}

\end{document}
