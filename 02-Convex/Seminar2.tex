\documentclass[12pt]{beamer}
\usepackage{mypres}
\usepackage[utf8]{inputenc}
\usepackage[russian]{babel}
\usepackage[T2A]{fontenc}

\expandafter\def\expandafter\insertshorttitle\expandafter{%
  \insertshorttitle\hfill%
  \insertframenumber\,/\,\inserttotalframenumber}
\title[Семинар 1]{Методы оптимизации. \\
Семинар 2. Выпуклые множества.}
\author{Александр Катруца}
\institute{Московский физико-технический институт,\\
Факультет Управления и Прикладной Математики} 
\date{12 сентября 2016 г.}
\begin{document}
\begin{frame}
\maketitle
\end{frame}

\begin{frame}{Напоминание}

\begin{itemize}
\item Предмет и задача курса
\item Общая постановка задачи математического программирования
\item Примеры задач оптимизации: 
\begin{itemize}
\item линейное программирование
\item метод наименьших квадратов
\item выпуклая оптимизация
\end{itemize}
\item Чем хороши выпуклые задачи?

\end{itemize}

\end{frame}

\begin{frame}{Выпуклые множества}

\end{frame}

\begin{frame}{Примеры}

\end{frame}

\begin{frame}{Афинные множества}

\end{frame}

\begin{frame}{Примеры}

\end{frame}

\begin{frame}{Конус}

\end{frame}

\begin{frame}{Примеры}

\end{frame}

\begin{frame}{Оболочки множеств}

\end{frame}

\begin{frame}{Примеры}

\end{frame}

\end{document}
