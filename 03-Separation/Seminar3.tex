\documentclass[12pt,russian]{beamer}
\usepackage{mypres}
\usepackage[utf8]{inputenc}
\usepackage[english,russian]{babel}
\usepackage[T2A]{fontenc}
\newcommand{\relint}{\mathbf{relint}}
\newcommand{\inter}{\mathbf{int}}

\expandafter\def\expandafter\insertshorttitle\expandafter{%
  \insertshorttitle\hfill%
  \insertframenumber\,/\,\inserttotalframenumber}
\title[Семинар 3]{Методы оптимизации. \\
 Семинар 3. Проекция точки на множество, отделимость, опорная гиперплоскость.}
\author{Александр Катруца}
\institute{Московский физико-технический институт,\\
Факультет Управления и Прикладной Математики} 
\date{19 сентября 2016 г.}

\begin{document}
\begin{frame}
\maketitle
\end{frame}

\begin{frame}{Напоминание}
\begin{itemize}
\item Аффинная оболочка и аффинное множество
\item Выпуклая оболочка и выпуклое множество
\item Коническая оболочка и выпуклый конус
\item Операции, сохраняющие выпуклость
\end{itemize}
\end{frame}

\begin{frame}{Внутренности множества}

\begin{block}{Внутренность множества}
Внутренность множества $G$ состоит из точек из $G$, таких что:
\[ 
\inter G = \{ \bx \in G \; | \; \exists \varepsilon > 0, B(\bx, \varepsilon) \subset G\},
\]
где $B(\bx, \varepsilon) = \{ \by \; | \; \| \bx - \by \| \leq \varepsilon \}$
\end{block}

\begin{block}{Относительная внутренность}
Относительной внутреностью множества $G$ называют следующее множество: 
\[
\relint G = \{ \bx \in G \; | \; \exists \varepsilon > 0,  B(\bx, \varepsilon) \cap \mathbf{aff} G \subseteq G \}
\]
\end{block}
Вопрос: зачем нужна концепция относительной внутренности?
\end{frame}

\begin{frame}{Примеры}

\end{frame}

\begin{frame}{Проекция точки на множество}
\small
\begin{block}{Расстояние между точкой и множеством}
Расстоянием $d$ от точки $\ba \in \bbR^n$ до замкнутого множества $X \subset \bbR^n$ по норме $\| \cdot \|$ является
\vspace{-4mm}
\[
d(\ba, X, \| \cdot \|) = \inf\{\| \ba - \by \| \; | \; \by \in X\}
\]
\end{block}
\begin{block}{Проекция точки на множество}
Проекцией точки $\ba \in \bbR^n$ на множество $X \subset \bbR^n$ по норме $\| \cdot \|$ будем называть такую точку $\pi_X(\ba) \in X$, что
\vspace{-4mm}
\[
\pi_X(\ba) = \argmin_{\by \in X} \| \ba - \by \|
\]
\end{block}
Вопросы: единственна ли проекция? Если нет, то в каких случаях единственна? Какая связь между единственностью проекции и выпуклостью множества?

\end{frame}

\begin{frame}{Факты о проекциях}

\begin{block}{Критерий проекции}
Точка $\pi_X(\ba) \in X$ является проекцией точки $\ba$ на множество $X$ $\Leftrightarrow$ $\| \ba - \bx \| \geq \| \ba - \pi_X(\ba) \|, \; \forall \bx \in X$.
\end{block}

\begin{block}{Критерий проекции для нормы $\ell_2$}
Точка $\pi_X(\ba) \in X$ является проекцией точки $\ba$ на множество $X$ $\Leftrightarrow$ $\langle \pi_X(\ba) - \ba, \bx - \pi_X(\ba) \rangle \geq 0, \; \forall \bx \in X$.
\end{block}
\end{frame}

\begin{frame}{Примеры}
\begin{itemize}
\item Проекция на шар $\{\bx \in \bbR^2 | \| \bx \|_* \leq 1\}$ в различных нормах
\item Проекция на аффинное множество $\{ \bx \in \bbR^n | \bA\bx = \mathbf{b}, \; \bA \in \bbR^{m \times n}, rank(\bA) = m \}$
\item Проекция на аффинное множество $\{ \bx \in \bbR^n | \bx = \bx_0 + \bS\by, \; \bS \in \bbR^{n \times m}, \; \by \in \bbR^m, rank(\bS) = m \}$
\end{itemize}
\end{frame}

\begin{frame}{Отделимость выпуклых множеств}
\small
\begin{block}{Определение}
Пусть $X_1, X_2 \subset \bbR^n$ произвольные множества. Они называются:
\vspace{-3mm}
\begin{itemize}
\item отделимыми, если $\exists \bp, \beta: \; \langle\bp, \bx_1 \rangle \geq \beta \geq \langle \bp, \bx_2\rangle$, $\forall \bx_1 \in X_1$ и $\forall \bx_2 \in X_2$.
\vspace{-3mm}
\item собственно отделимыми, если они отделимы и $\exists \bx_1^* \in X$ и $\exists \bx_2^* \in X$: $\langle \bp, \bx_1^* \rangle > \langle \bp, \bx_2^* \rangle$
\vspace{-3mm}
\item сильно отделимыми, если $\exists \bp \neq 0$ и $\beta$: $\inf\limits_{\bx_1 \in X_1} \langle \bp, \bx_1 \rangle > \beta > \sup\limits_{\bx_2 \in X_2} \langle \bp, \bx_2 \rangle$
\vspace{-3mm}
\item строго отделимы, если $\forall \bx_1 \in X_1$ и $\forall \bx_2 \in X_2$: $\langle \bp, \bx_1 \rangle > \langle \bp, \bx_2 \rangle$.
\end{itemize} 
\end{block}

\begin{block}{Разделяющая гиперплоскость}
Разделяющей гиперплоскостью для множеств $X_1, X_2$ является такая гиперплоскость $\{ \bx | \langle \bp, \bx \rangle = \beta \}$, что $\langle \bp, \bx_1 \rangle \geq \beta$ для всех $\bx_1 \in X_1$ и $\langle \bp, \bx_2 \rangle \leq \beta$ для всех $\bx_2 \in X_2$   
\end{block}
\end{frame}

\begin{frame}{Факты об отделимости}
\begin{block}{Существование}
Пусть $X_1$ и $X_2$ выпуклые непересекающиеся множества, тогда существует разделяющая их гиперплоскость. 
\end{block}

\begin{block}{Критерий для выпуклых множеств}
Два выпуклых множества, таких что по крайней мере одно из них открыто, не пересекаются тогда и только тогда, когда существует разделяющая гиперплоскость.
\end{block}

\begin{block}{Критерий сильной отделимости}
Два выпуклых множества сильно отделимы тогда и только тогда когда расстояние между ними положительно.
\end{block}

\end{frame}

\begin{frame}{Примеры}
\begin{itemize}
\item Построить разделяющую гиперплоскость для множеств $X_1, X_2$: $X_1 = \{ (x_1, x_2) \in \bbR^2 | x_1x_2 > 1, x_1 > 0 \}$, $X_2 = \{ (x_1, x_2) \in \bbR^2 |  x_2 \leq 9 + \frac{4}{x_1 - 1} \}$.
\item Критерий разрешимости системы строгих линейных неравенств $\bA\bx \leq \mathbf{b}$ как не пересечение аффинного множества $\{\mathbf{b} - \bA\bx | \bx \in \bbR^n \}$ и множества $\{ \by \in \bbR^m | y_i > 0 \}$ 
\item Пример двух замкнутых выпуклых не пересекающихся множеств, которые не являются строго отделимыми
\item Построить разделяющую гиперплоскость для множеств $X_1 = \{ \bx \in \bbR^n | \| \bx \|^2_2 \leq 1 \}$ и $X_2 = \{ \bx \in \bbR^n | x_1^2 + \ldots + x_{n-1}^2 + 1 \leq x_n \}$.
\end{itemize}
\end{frame}
\begin{frame}{Опорная гиперплоскость}
\begin{block}{Опорная гиперплоскость}

\end{block}

\begin{block}{Теорема об опорной гиперплоскости}
В любой граничной (относительно граничной) точке выпуклого множества существует опорная (собственно опорная) гиперплоскость.
\end{block}
\end{frame}


\begin{frame}{Примеры}

\end{frame}


\begin{frame}{Резюме}
\begin{itemize}
\item Внутренность и относительная внутренность выпуклого множества
\item Проекция точки на множство
\item Отделимость выпуклых множеств
\item Опорная гиперплоскость
\end{itemize}
\end{frame}



\end{document}

http://math.stackexchange.com/questions/26456/disjoint-convex-sets-that-are-not-strictly-separated