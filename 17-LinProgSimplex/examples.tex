\documentclass[12pt]{article}
\usepackage[utf8]{inputenc}
\usepackage[russian]{babel}
\usepackage[T2A]{fontenc}
\usepackage{amsmath,amssymb,amsfonts}
\usepackage[top=2cm,bottom=2cm,left=2cm,right=2cm]{geometry}
\newcommand{\bx}{\mathbf{x}}
\newcommand{\bA}{\mathbf{A}}
\usepackage{indentfirst}

\title{Примеры решения задач линейного программирования симплекс-методом}
\author{Александр Катруца}
\date{}

\begin{document}
\maketitle
\thispagestyle{empty}
Здесь использованы материалы из книги~\cite{intro2lin}. 
\begin{enumerate}
\item Решить задачу табличным симплекс методом:
\begin{equation*}
\begin{split}
& \min_{\bx} -10x_1 - 12x_2 - 12x_3\\
\text{s.t. } & x_1 + 2x_2 +2x_3 \leq 20\\
& 2x_2 + x_2 + 2x_3 \leq 20\\
& 2x_1 + 2x_2 + x_3 \leq 20\\
& x_{1,2,3} \geq 0
\end{split}
\end{equation*}
\textbf{Решение:} по виду задачи ясно, что она не в канонической  форме. Введём дополнительные переменные и запишем её в канонической форме:
\begin{equation*}
\begin{split}
& \min_{\bx} -10x_1 - 12x_2 - 12x_3\\
\text{s.t. } & x_1 + 2x_2 +2x_3 + x_4 = 20\\
& 2x_2 + x_2 + 2x_3 + x_5 = 20\\
& 2x_1 + 2x_2 + x_3 + x_6 = 20\\
& x_{1,2,3,4,5,6} \geq 0
\end{split}
\end{equation*}
Заметим, что матрица $\bA \in \mathbb{R}^{m \times n}$, где $m=3$ и $n=6$.
Теперь нужно найти угловую точку допустимого множества, то есть такую точку, чтобы она лежала в множестве и существовало множество индексов $\mathcal{B} \subset \{1, \dots, n\}$ мощностью $|\mathcal{B}| = m = 3$, что матрица из столбцов матрицы $\bA$ с индексами из множества $\mathcal{B}$ была невырождена, и координаты угловой точки с индексами не из множества $\mathcal{B}$ были нулевыми.
В данном случае достаточно очевидно, что $\bx_0 = (0, 0, 0, 20, 20, 20)$, $\mathcal{B}_0 = \{4, 5, 6 \}$ и матрица базиса $\mathbf{B}_0 = \mathbf{I}_m$~--- невырождена.
Если начальная угловая точка не так очевидна, необходимо выполнить двухфазный симплекс-метод или M-метод. 
Такой пример будет приведён ниже.

Теперь составим таблицу симплекс-метода, модифицируя которую получим решение поставленной задачи.

\begin{table}[!ht]
\centering
\begin{tabular}{|c|cccccc|}
\hline
& $x_1$ & $x_2$ & $x_3$ & $x_4$ & $x_5$ & $x_6$\\
\hline
$-\mathbf{c}_{\mathcal{B}_0}^{\top}\bx_{\mathcal{B}_0} = 0$ & $-10$ & $-12$ & $-12$ & $0$ & $0$ & $0$ \\
\hline
$x_4 = 20$ & 1 & 2 & 2 & 1 & 0 & 0 \\
$x_5 = 20$ & 2 & 1 & 2 & 0 & 1 & 0 \\
$x_6 = 20$ & 2 & 2 & 1 & 0 & 0 & 1 \\
\hline
\end{tabular}
\end{table}
\end{enumerate}

\begin{thebibliography}{9}

\bibitem{intro2lin} 
Dimitris Bertsimas and John N. Tsitsiklis. \emph{Introduction to linear optimization}, Belmont, MA: Athena Scientific, 1997, 5th edition

\end{thebibliography}
\end{document}

