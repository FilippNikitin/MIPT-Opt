\documentclass[12pt,russian]{beamer}
\usepackage{mypres}
\usepackage[utf8]{inputenc}
\usepackage[T2A]{fontenc}
\usepackage[russian]{babel}
\newcommand{\relint}{\mathbf{relint}}
\newcommand{\inter}{\mathbf{int}}

\expandafter\def\expandafter\insertshorttitle\expandafter{%
  \insertshorttitle\hfill%
  \insertframenumber\,/\,\inserttotalframenumber}
\title[Семинар 9]{Методы оптимизации. \\
 Семинар 9. Условия оптимальности.}
\author{Александр Катруца}
\institute{Московский физико-технический институт,\\
Факультет Управления и Прикладной Математики} 
\date{31 октября 2016 г.}

\begin{document}
\begin{frame}
\maketitle
\end{frame}

\begin{frame}{Напоминание}
\begin{itemize}
\item Конус возможных направлений
\item Касательный конус
\item Острый экстремум
\end{itemize}
\end{frame}

\begin{frame}{Мотивация}

\begin{block}{Вопрос 1}
Как проверить, что точка является решением оптимизационной задачи? 
\end{block}

\begin{block}{Вопрос 2}
Из каких условий можно найти решение оптимизационной задачи?
\end{block}
\end{frame}

\begin{frame}{Существование решения}

\end{frame}

\begin{frame}{Условия оптимальности}
\begin{block}{Определение}
Условием оптимальности будем называть некоторое выражение, выполнимость которого даёт необходимое и (или) достаточное условие экстремума. 
\end{block}
Классы задач:
\begin{itemize}
\item Общая задача минимизации
\item Задача безусловной минимизации
\item Задача минимизации с ограничениями типа равенств
\item Задача минимизации с ограничениями типа равенств и неравенств
\end{itemize}
\end{frame}

\begin{frame}{Общая задача минимизации}

\begin{block}{Задача}

\end{block}

\begin{block}{Критерий оптимальности}

\end{block}

\end{frame}

\begin{frame}{Примеры}

\end{frame}

\begin{frame}{Задача безусловной минимизации}

\begin{block}{Задача}

\end{block}

\begin{block}{Критерий оптимальности}

\end{block}

\end{frame}

\begin{frame}{Примеры}

\end{frame}

\begin{frame}{{\small Задача минимизации с ограничениями типа равенств}}

\begin{block}{Задача}

\end{block}

\begin{block}{Критерий оптимальности}

\end{block}

\end{frame}

\begin{frame}{Примеры}

\end{frame}

\begin{frame}{{\small Задача минимизации с ограничениями типа равенств и неравенств}}

\begin{block}{Задача}

\end{block}

\begin{block}{Критерий оптимальности}

\end{block}

\end{frame}

\begin{frame}{Примеры}

\end{frame}

\end{document}